\section{Metodologia}\label{sec:metodo}

\subsection*{Organização dos códigos}

O programa foi separado em três arquivos: \begin{inparaenum}[\itshape i)\upshape]
\item um binário em \emph{Python} que realiza tarefas gerais da linha de comando (\emph{m21});
\item rotinas do Music21 (\emph{m21utils.py}); e 
\item um para rotinas externas (\emph{tools.py})
\end{inparaenum}\footnote{Todos códigos, exemplos e documentação estão disponíveis \url{https://www.github.com/jahpd/m21}.}.

\subsection*{Categorização do software}

Nas palavras de \cite[p.~x-xiii]{cope_prefacio_2008}, o \emph{m21} pode ser classificado como uma ferramenta para uma Assistência Gerada por Computador (\emph{Computer Generated Assistance} ou CGA). Dentro das sub-categorias de CGA propostas por Cope, o \emph{m21} pode ser incuído nos três modos abaixo:\begin{inparaenum}[\itshape 1)\upshape]
\item uso de uma Linguagem de Programação em texto (PLs)(\emph{Programming Languages}) ao invés de uma linguagem de programação visual (VPL); 
\item o material partitural é gerado para performance humana ao invés de uma performance eletroacústica;
\item utilização de regras diversas (\emph{Rules Based}) e composição dirigida por dados (\emph{Data-Driven}).
\end{inparaenum}

\subsection*{Método de composição}

Em geral, o procedimento de composição se deu a partir da execução de um comando \emph{m21}, como explicitado nas seções \ref{sec:m21} e \ref{sec:resultados}.

O passos executados pelo comando, no exemplo apresentado na seção \ref{sec:resultados}, foram \begin{inparaenum}[\itshape i)\upshape]:x
\item subtração de compassos, aleatoriamente;
\item dos compassos restantes, notas pertencentes a este compasso foram suprimidas em um bloco harmônico; e
\item destes blocos, as oitavas serão embaralhadas para gerar simultanóides \cite{koellreutter_introducao_1987};
\end{inparaenum}

O material resultante do processo acima foi editado no \cite{musescore_2015}. Algumas interferências não previstas foram incluídas. Seguimos com a observação de ``organicidades'' próprias do material pré-composicional: fraseados, pontos culminantes e pontos de relaxamento, criando um pequeno discurso musical.

Por último, a peça foi diagramada no \cite{lilypond_2015}.