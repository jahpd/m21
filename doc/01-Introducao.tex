\section{Introdução}

Este artigo trata de um protótipo, \emph{m21.py},  desenvolvido a partir da biblioteca \cite{music21_2015}, para composição e análise musical. Na seção \ref{sec:trabalhos} contextualizo o que me levou a elaborar o programa.  Na seção \ref{sec:metodo} descrevo as tarefas realizadas para desenvolvimento do \emph{m21}. Na seção \ref{sec:m21} a utilização do \emph{m21}. 

Este código capacitou a produção de um número considerável de exercícios composicionais para piano. A intenção é oferecer uma ferramenta quase imediatista de material pré-composicional para exercícios criativos. Pode ser útil em cursos de Composição Assistida por Computador, em universidades ou em oficinas de arte. Por economia de espaço,apresentarei na seção \ref{sec:resultados} um exemplo. Na seção \ref{sec:conclusao} discuto problemas técnicos do \emph{software}, problemas composicionais. Na seção 7 planos futuros.