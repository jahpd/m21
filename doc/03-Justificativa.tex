\section{Justificativa}\label{sec:justificativa}

Este trabalho capacitou a produção de um número considerável de exercícios criativos, para piano solo. A intenção é oferecer uma ferramenta para geração de materiais pré-composicionais, de maneira quase imediatista. Pode ser útil em cursos de Composição Assistida por Computador, em universidades ou em oficinas de arte. 

A programação de um algoritmo de erro (ver seção \ref{sec:m21}) foi um processo de aprendizagem da linguagem \emph{Python}, para programar peças pós-tonais de maneira didática. Isto concorda com  o discurso de Cascone (é sugerido transpormos o termo ``processamento de sinais digitais'' para ``harmonia''):
\ \\
\begin{quote}
Porque as ferramentas usadas neste estilo de música incorporam conceitos avançados de processamento de sinal digital, a utilização das ferramentas por artistas glitch tendem ser baseadas mais na experimentação do que em uma investigação empírica. Desta forma, usos não-intencionais se tornaram uma segunda permissão garanida. Dizem que alguém não necessita de treino para para usar programas de processamento de sinais - apenas ``fuçar'' até obtermos o resultado desejado. Algumas vezes, não conhecer a operação teórica da ferramenta pode resultar em casos mais interessantes, por ``pensar fora da caixa''\cite[p.~397]{cascone_aesthetics_2000}\footnote{Tradução de \emph{Because the tools used in this style of music embody advanced concepts of digital signal processing, their usage by glitch artists tends to be based on experimentation rather than empirical investigation. In this fashion, unintended usage has become the second permission granted. It has been said that one does not need advanced training to use digital signal processing programs-just "mess around" until you obtain the desired result. Sometimes, not knowing the theoretical operation of a tool can result in more interesting results by "thinking outside of the box.''}}
\end{quote}