\section{Trabalhos relacionados}\label{sec:trabalhos}

Durante uma pesquisa de mestrado, sobre \emph{live coding}, tive contato com o Music21 que, segundo \cite{soares_luteria_2015}:

\begin{quote}
É uma biblioteca projetada para trabalhar com manipulação e análise de corpus de arquivos partituráveis. Prepara a conversão entre diversos arquivos de dados musicais. (\ldots) Music21 tem uma abordagem voltada para uma "musicologia assistida por computador" e já tem incorporada em suas classes algumas ferramentas comuns a esta prática como: numeração de grau funcional de acorde, numeração de classes de altura usando a classificação de Allen Forte : a implementação dos algoritmos de detecção de tonalidade elaborado por Krumhansl (1990) e aperfeiçoada por Temperley (2001), busca de padrões como transposições e inversões e outros.\cite[p.~71-72]{soares_luteria_2015}
\end{quote}

Ao invés de compor adicionando informações aos dados musicais, busquei na \emph{Estética do Erro} \cite{cascone_aesthetics_2000} os procedimentos básicos para composição, explicados com mais detalhes na seção \ref{sec:metodo}. Resultados sonoros procedentes da colagem e do erro dependem muito do \emph{input}. Aplicar o mesmo algoritmo de erro para diferentes materiais, ou, aplicar diferentes erros para um único material, não resulta em um produto homogêneo.

Busquei então utilizar diferentes documentos do \emph{corpus}, e um mesmo erro para realizar um tipo de música que, segundo \cite[p.~18]{soares_luteria_2015}, existia intensamente ``antes da preocupação imediata com os timbres ou da era das manipulações de amostras sonoras - e de certa maneira ainda proto-serialista. Uma música por vezes chamada politonal, polimodal ou usando o termo de Straus (2004): pós-tonal.''. Sendo um tipo de composição que não é nova, mas relevante do ponto de vista histórico e didático-composicional,  busquei elaborar uma ferramenta para ser usada em processos criativos musicais. 

\subsection*{Questões pessoais}

Não deixo de mencionar uma antiga conversa com o compositor Franscisco Zmekhol Nascimento de Oliveira, que levou-me a compor segundo regras arbitrárias, na contingência do momento. Isto é, uma música para cada dia da existência. 

Ao realizar o mesmo procedimento de composição de um mesmo documento do \emph{corpus} bachiano, o material pré-composicional resultante deve ser diferente de qualquer outro. 

Para realizar tecnicamente, valores não-determinados em operações determinísticas são utilizados.

Para compor-improvisar com os materiais resultantes, apliquei princípios de articulação do som pelo silêncio, elaborados por rascunhos de partituras-planimétricas, utilizadas por \cite{koellreutter_introducao_1987}.